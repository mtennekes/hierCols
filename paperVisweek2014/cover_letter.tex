\documentclass{letter}
\addtolength{\medskipamount}{-.75\medskipamount}
\usepackage{hyperref}
\signature{Martijn Tennekes and Edwin de Jonge}
\address{Statistics Netherlands \\ P.O. Box 4481 \\ Heerlen \\ the Netherlands}
\begin{document}
\begin{letter}{InfoVis 2014 \\ dr. M. Tory, University of Victoria \\ dr. J. Heer, University of Washington \\ prof. H. Hauser, University of Bergen}
\opening{Dear dr. Tory, dr. Heer, and prof. Hauser,}

We would like to thank you for the conditionally acceptance of our manuscript for publication in the IEEE TVCG. Furthermore, we would like express our thankfulness towards the reviewers whose profound and insightful feedback enabled us to improve our manuscript substantially. In this letter, we summarize the changes we made to the manuscript.


First, we fixed all grammatical issues and typos addressed by the reviewers. Obviously, we rechecked to the whole document carefully on remaining grammatical errors and typos.

Next, we improved the paper regarding all the issues listed by the primary reviewer:

\begin{itemize}
\item We corrected all misleading terms addressed by the reviewers. In the revised manuscript we use the term node-link diagram where we incorrectly used the term graph. Further, we use the term labels rather than codes which has obviously a different meaning in computer science.
\item We briefly addressed the role of tree coloring in the introduction (first paragraph), and also discussed it in the discussion.
\item We better explained and clarified the algorithm: with respect to the hue fraction (last paragraph 3.1), and the permutation step (second paragraph 3.1.1)
\item Obviously, we were unable to set up another user study with more advanced tasks within this review cycle. However, we think our user study still showed some insightful results, which is also mentioned in Review 1.  As this reviewer suggested, we briefly summarized the user study results in the introduction (second last paragraph). Furthermore, we extended the user study discussion in section 7 (5th, 6th paragraph). We recommended additional user studies as suggested by Review 3, for further research.
\item Color vision deficiency is indeed a very important aspect of information visualization. Altough we took people with color deficiencty into account in our user study (we also took
questionnairres from them, which we now mentioned in section 5.2, first paragraph),  we neglected to analyse Tree Colors in this context. We added a section of color vision deficiency (3.4) and added a paragraph in the discussion (section 7, second last paragraph).
\item Related work
\item We took care of all presentation problems addressed by Review 2
\end{itemize}

Other issues that were suggested:

Review 3: we described the permutation step of the algorimth better, and illustrated it with a short example. The reason to chose 144 degrees over the golden angle of 137,5 degrees is that in a the case of 5 siblings, the angle will be rounded down to 72, resulting in the unwanted permutation order of [1,2, 3, 4, 5]. Of course, we could make this case a special case like the 3 and 4 case.







\closing{Sincerely,}
\end{letter}
\end{document}
