% $Id: template.tex 11 2007-04-03 22:25:53Z jpeltier $

%\documentclass{vgtc}                          % final (conference style)
\documentclass[review]{vgtc}                 % review
%\documentclass[widereview]{vgtc}             % wide-spaced review
%\documentclass[preprint]{vgtc}               % preprint
%\documentclass[electronic]{vgtc}             % electronic version

%% Uncomment one of the lines above depending on where your paper is
%% in the conference process. ``review'' and ``widereview'' are for review
%% submission, ``preprint'' is for pre-publication, and the final version
%% doesn't use a specific qualifier. Further, ``electronic'' includes
%% hyperreferences for more convenient online viewing.


\usepackage{mathptmx}
\usepackage{graphicx}
\usepackage{times}

\onlineid{0}

%% declare the category of your paper, only shown in review mode
\vgtccategory{Research}

%% allow for this line if you want the electronic option to work properly
\vgtcinsertpkg

%% In preprint mode you may define your own headline.
%\preprinttext{To appear in an IEEE VGTC sponsored conference.}

%% Paper title.

\title{Hierarchical qualitative color palettes}

\author{Martijn Tennekes\thanks{e-mail: m.tennekes@cbs.nl}\\ %
        \scriptsize Statistics Netherlands %
\and Edwin de Jonge\thanks{e-mail:e.dejonge@cbs.nl}\\ %
     \scriptsize Statistics Netherlands}

\abstract{ Color is an important means to display categorical data in statistical graphics. Categories are often hierarchically structured in a classification tree, but most qualitative color palettes do not take this hierarchy into account.  

We present a method to map tree structures to colors from the Hue-Chroma-Luminance (HCL) color model. The HCL color space, which is a transformation of the CIELUV color space, is known for its well balanced perceptual properties. Our experiments suggest that hierarchical qualitative color palettes are very useful: not only for improving standard hierarchical visualizations such as trees and tree maps, but also for showing tree structure in non-hierarchical visualizations.
}

\CCScatlist{ 
 \CCScat{H.5.2}{ Information Interfaces and Presentation}%
{User Interfaces}{User-centered design}
}

%% Copyright space is enabled by default as required by guidelines.
%% It is disabled by the 'review' option or via the following command:
% \nocopyrightspace

\graphicspath{{../plots/}}

\begin{document}

\firstsection{Introduction}

\maketitle

Hierarchical data are of crucial importance in official statistics. Most official data are published using hierarchically structured categories, for instance geographic regions or economic activities. A recent trend in data analysis practice, is to follow a top-down approach rather than to check  each individual record of a survey or administrative source. Several data visualization methods are useful to explore and analyze hierarchical statistical data, for instance treemaps
[Shneiderman hier noemen..., plus jezelf...] \cite{,tennekes2011b}. Color palettes reflecting the  hierarchical structure would be very useful in supporting visual analysis.

Assigning colors to categories is far from trivial. On the one hand, qualitative colors should be distinct, but on the other hand they should not suggest non-existent order or proximity and introduce perceptual bias. The selection of color palettes for categorical data first depends on the type of data. For nominal data, such as gender or nationality, qualitative color palettes are used, while for ordinal data, such as level of urbanization, sequential or diverging palettes are used \cite{brewer03, zeileis2009}. However, for hierarchical categories there are no specific guidelines for selecting color palettes, to the best of our knowledge.

Although many tree visualizations are proposed in literature \cite{schulz2011}, most of them use color to a small extent. A visualization technique that uses color as a major attribute is the InterRing \cite{yang2002}, a navigation tool with a radial layout. The leaf nodes are assigned to a different hue values. The color of a parent node is derived from averaging the colors of its children, where larger branches have more weight. An implicit effect is that colors of higher hierarchical levels are less saturated, except for one-child-per-parent branches.


\section{Method}

Our method maps a tree structure on colors in HCL space, such that is reflects it hierarchical properties. The Hue-Chroma-Luminance (HCL) space, which is a transformation of the CIELUV color space, is designed with the aim to control human color perception~\cite{ihaka2003}.
Colors with different hue values are perceptually uniform in colorfulness and brightness, which does not hold for the popular Hue-Saturation-Value (HSV) color space~\cite{zeileis2009}. The hue $H$ takes values from 0 to 360, and the chroma $C$ and luminance $L$ take values from 0 to 100.

We use $H$ for the tree structure, where the hue of child nodes resemble the hue of their parents. $C$ and $L$ are used to discrimate the different hierarchical levels.

In this paper we illustrate our method with the European classification system of economic activity NACE. (reference?). This is a rather large classification system, (iets zeggen over de grootte?). Figure~\ref{fig:sbiF} show a small part of the NACE: section F (Construction). Figure~\ref{fig:sbiF} is a typical tree visualization using a radial layout, but is improved by using nodes with colors from our hierarchical color palette. 

\begin{figure}[htb]
  \centering
  \includegraphics[width=3.5in]{sbi_F.pdf}
  \caption{Tree structure of economic sector F of NACE.}\label{fig:sbiF}
\end{figure}

For selecting hue values we use the following recursive algorithm. It will assign to each node $v_i$ of a tree structure a hue value $h_i$ and a hue value range $r_i$.

We start with the root node, which has by default hue range $[0, 360]$:

{\bf AssignHue($v$, $r$)}
\begin{enumerate} \itemsep1pt \parskip0pt 
\parsep0pt
\item assign the middle hue value in $r$ to $h$ \footnote{in most cases the root node itself won't be drawn.}
\item $N$ is number of child nodes of $v_i$, if $N>0$ :
\begin{enumerate} \itemsep1pt \parskip0pt 
\parsep0pt
\item divide $r$ in $N$ equal parts $r_i$;
\item permute the $r_i$'s and assign them to the child nodes
\item reduce each $r_i$ by keeping the middle fraction $f$;
\item for each child nodes $v_i$ DO AssignHue($v_i$,$r_i$)
\end{enumerate}
\end{enumerate}

\begin{figure}[htb]
  \centering
  \includegraphics[width=3.5in]{hcl_method.pdf}
  \caption{Assignment of hue values.}\label{fig:wheel}
\end{figure}

This division of the hue range is illustrated in Figure~\ref{fig:wheel}: in (a) the full hue range (for a constant $C=60$ and $L=70$)  is divided among the three children of the root, in (b) the middle parts are kept, in (c) and (d) these steps are recursively taken for the deepest two hierarchical layers.

Ad 2(b) In most hierarchical structures, there is no order between siblings. When the nodes in such structure are plotted in a linear or radial layout, the colors of the siblings should not introduce a perceptual order. Therefore, the assigned hue ranges are permuted among the siblings. The used permutation order is based on the five-elements-permutation $[1, 3, 5, 2, 4]$. Furthermore, the permutation within even numbered branches is reversed to differentiate between branches. Note the labeling of the color wheel that shows that the assignment of colors is permuted.

Ad 2(c) The fraction is needed to introduce a `hue gap' between nodes with a different parent. The fraction $f$ is by default set to $0.75$. This choice is a trade-off between discriminating different main branches and discriminating different leaf nodes. 

In order to show depth, we let $C$ and $L$ values only depend on the depth of the corresponding nodes. We let the $L$ decrease linearly with depth and $C$ increase: having more intense colors helps in discriminating leaf nodes.

\begin{figure}[htb]
  \centering
  \includegraphics[width=3.5in]{treemap_F.pdf}
  \caption{Treemap with hierarchical colors}\label{fig:treemapF}
\end{figure}



\begin{figure}[htb]
  \centering
  \includegraphics[width=1.75in]{bar_chart.pdf}
  \caption{Bar chart with hierarchical colors}\label{fig:barchart}
\end{figure}
\begin{figure}[htb]
  \centering
  \includegraphics[width=1.75in]{stackedbar_chart.pdf}
  \caption{Stacked bar chart with hierarchical colors}\label{fig:barchart}
\end{figure}



\section{Application}
The hierarchical colors can be applied to enhance standard tree visualizations, as we saw in . figure~\ref{fig:sbiF}. Strictly speaking this is redundant color usage, but in our opinion
it can improve many tree visualizations. A second example of improvement is depicted in figure~\ref{fig:treemapF}. It shows a treemap depicting (fictious) turnover in Construction (NACE F). In official statistics, turnover is available for each business enterprise in a business register, and aggregated according to the NACE tree. The colors of the smallest rectangles correspond to the most detailed NACE layer, while the color of higher NACE layers is used to for the text label backgrounds. This treemap is created with the free and open source R package treemap \cite{treemap}.

Hierarchical colors can also improve visualizations without explicit tree structure. The colors hint at the underlying tree structure.
As an example of a non-hierarchically-structured plot, a bar chart of random data is depicted in ~\ref{fig:barchart}. Such graphics could be useful when the hierarchical structure will not be the main focus in the conducted analyses. ....

\section{Further research}

We recommend to further investigate the proposed hierarchical color method, and to evaluate the obtained color palettes in various statistical graphics. This should include a user study in the effectiveness of this hierarchical color palette.

%\acknowledgements{
%The authors wish to thank A, B, C. This work was supported in part by
%a grant from XYZ.}

\bibliographystyle{abbrv}
%%use following if all content of bibtex file should be shown
%\nocite{*}
\bibliography{hiercolor}
\end{document}
